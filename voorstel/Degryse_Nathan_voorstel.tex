%==============================================================================
% Sjabloon onderzoeksvoorstel bachelorproef
%==============================================================================
% Gebaseerd op LaTeX-sjabloon ‘Stylish Article’ (zie voorstel.cls)
% Auteur: Jens Buysse, Bert Van Vreckem
%
% Compileren in TeXstudio:
%
% - Zorg dat Biber de bibliografie compileert (en niet Biblatex)
%   Options > Configure > Build > Default Bibliography Tool: "txs:///biber"
% - F5 om te compileren en het resultaat te bekijken.
% - Als de bibliografie niet zichtbaar is, probeer dan F5 - F8 - F5
%   Met F8 compileer je de bibliografie apart.
%
% Als je JabRef gebruikt voor het bijhouden van de bibliografie, zorg dan
% dat je in ``biblatex''-modus opslaat: File > Switch to BibLaTeX mode.

\documentclass{hogent-article}

\usepackage{lipsum}

%------------------------------------------------------------------------------
% Metadata over het voorstel
%------------------------------------------------------------------------------

%---------- Titel & auteur ----------------------------------------------------

% TODO: geef werktitel van je eigen voorstel op
\PaperTitle{ARM: van mobiele toestellen en IoT naar de desktop}
\PaperType{Onderzoeksvoorstel Bachelorproef 2021-2022} % Type document

% TODO: vul je eigen naam in als auteur, geef ook je emailadres mee!
\Authors{Nathan Degryse\textsuperscript{1}} % Authors
\CoPromotor{Mathieu Audenaert\textsuperscript{2}}
\affiliation{\textbf{Contact:}
  \textsuperscript{1} \href{mailto:nathan.degryse@student.hogent.be}{nathan.degryse@student.hogent.be};
  \textsuperscript{2} \href{mailto:mathieu.audenaert@hotmail.com}{mathieu.audenaert@hotmail.com};
}

%---------- Abstract ----------------------------------------------------------

\Abstract{ARM-processoren worden grotendeels gebruikt voor mobiele apparaten zoals smartphones, tablets en allerhande Internet of Things (IoT) apparaten. Aan de andere kant situeert zich de traditionele en welbekende x86-64 processorarchitectuur die dominant is in de wereld van laptops, desktops en servers. Sinds enkele jaren is er een opvallende trend in opmars, zijnde het gebruik van ARM-processoren in traditionele laptops en desktops gericht op zowel consumenten als professionals. Het doel van dit onderzoek is om na te gaan of het hedendaagse landschap van desktop besturingssystemen en software klaar is voor de nakende coëxistentie van ARM en x86-64 processoren. Hierbij is het belangrijk om stil te staan bij de belangrijkste kenmerken van de twee architecturen. Deze hebben uiteraard allebei voor- en nadelen, maar in de context van dit onderzoek is het de bedoeling om stil te staan bij kenmerken zoals software compatibiliteit, gebruiksvriendelijkheid en prestatie per watt. Het belang van deze laatste eigenschap valt uiteraard niet te onderschatten in een wereld die steeds meer aandacht heeft voor verbruik en duurzaamheid. Het praktische luik van dit onderzoek omvat een opstelling met scenario's waarin wordt onderzocht in welke mate het haalbaar is voor hedendaagse professionals in verschillende domeinen om over te schakelen op een toestel met een desktop besturingssysteem die gebruikt maakt van ARM technologie. Een bijkomend luik van dit praktische onderzoek is het staven van een van de grootste voordelen aan het gebruik van dezelfde processortechnologie in verschillende klassen van toestellen, zijnde het maken van een ‘proof-of-concept’ applicatie die uitvoerbaar is op zowel smartphones, tablets als desktops.
}

%---------- Onderzoeksdomein en sleutelwoorden --------------------------------
% TODO: Sleutelwoorden:
%
% Het eerste sleutelwoord beschrijft het onderzoeksdomein. Je kan kiezen uit
% deze lijst:
%
% - Mobiele applicatieontwikkeling
% - Webapplicatieontwikkeling
% - Applicatieontwikkeling (andere)
% - Systeembeheer
% - Netwerkbeheer
% - Mainframe
% - E-business
% - Databanken en big data
% - Machineleertechnieken en kunstmatige intelligentie
% - Andere (specifieer)
%
% De andere sleutelwoorden zijn vrij te kiezen

\Keywords{Onderzoeksdomein. Systeembeheer --- processorarchitectuur --- besturingssystemen --- virtualisatie} % Keywords
\newcommand{\keywordname}{Sleutelwoorden} % Defines the keywords heading name

%---------- Titel, inhoud -----------------------------------------------------

\begin{document}

\flushbottom % Makes all text pages the same height
\maketitle % Print the title and abstract box
\tableofcontents % Print the contents section
\thispagestyle{empty} % Removes page numbering from the first page

%------------------------------------------------------------------------------
% Hoofdtekst
%------------------------------------------------------------------------------

% De hoofdtekst van het voorstel zit in een apart bestand, zodat het makkelijk
% kan opgenomen worden in de bijlagen van de bachelorproef zelf.
%---------- Inleiding ---------------------------------------------------------

\section{Introductie en State-of-the-art} % The \section*{} command stops section numbering
\label{sec:introductie}

\subsection{Wat is een ARM processor?}
Het acroniem ARM wordt gebruikt om te refereren naar de processorarchitectuur alsook het bedrijf die verantwoordelijk is voor de ontwikkeling van deze architectuur. Initieel stond ARM voor Acorn RISC Machine, maar dit werd later aangepast naar Advanced RISC Machines. Het bedrijf werd officieel opgericht in 1990 als een samenwerkingsverband tussen Acorn Computers, Apple Inc. en VLSI Technology \autocite{Walshe2015}. Jaarlijks worden er zo een tien miljard ARM processors verkocht waarvan de grootste groep wordt gebruikt in mobiele toestellen zoals smartphones en tablets. Dit is echter niet de enige afzetmarkt. ARM processoren zijn terug te vinden in zowat elke categorie van apparaten, gaande van smartwatches tot servers \autocite{Harris2015}. ARM verleent, in tegenstelling tot andere chipfabrikanten zoals Intel en AMD licenties om processoren te maken als onderdeel van een system-on-a-chip (SoC). Dit houdt in dat technologiereuzen, zoals Apple en Qualcomm, ARM processoren ontwikkelen op basis van gekochte licenties of op basis van coöperatie \autocite{Asghar2020}. Zoals de naam doet vermoeden maken ARM processoren gebruik van de Reduced Instruction Set Computer (RISC) architectuur. Dit in tegenstelling tot x86 processoren die gebruikmaken van de Complex Instruction Set Computer (CISC) architectuur \autocite{Harris2015}.

\subsection{Waarom kiezen voor ARM processoren?}
Traditionele x86 processoren die gebruikt worden in desktop en server toestellen liggen aan de basis van het torenhoge energieverbruik van datacenters. Dit hoge energieverbruik is niet enkel te wijten aan de CISC architectuur, maar ook aan de apparatuur die nodig is om de processoren af te koelen. Aan de andere kant situeren zich ARM processoren die een goede balans leveren tussen performantie en energieverbruik. Deze relatie wordt vaak uitgedrukt als prestatie per watt. Zoals eerder vermeld maken ARM processoren gebruik van de RISC architectuur. RISC instructies zijn eenvoudiger van ontwerp dan hun CISC tegenhangers. Het gevolg daarvan is dat er meer RISC instructies nodig zijn om eenzelfde taak uit te voeren. Dit houdt uiteraard in dat het geheugenverbruik van het systeem groter is \autocite{Aroca2012}. Een andere typerende eigenschap van RISC instructies is dat ze worden uitgevoerd in één klokcyclus. Als gevolg is er sprake van minder latentie en een kortere systeem responstijd. ARM processoren zijn vaak terug te vinden in de vorm van een system-on-a-chip (SoC). Dit houdt in dat meerdere functies op één chip worden geïntegreerd. Een gevolg daarvan is dat een systeem minder componenten heeft en bijgevolg kleiner is van omvang. Additioneel leidt deze eigenschap er ook toe dat het systeem goedkoper is om te produceren en een lager energieverbruik heeft \autocite{Ravali2016}.

\subsection{Implicaties bij het gebruik van de ARM architectuur}
Zoals eerder vermeld maken ARM en x86 processoren gebruik van een verschillende instructieset. Dit houdt in dat programma’s worden ontworpen met een specifieke instructieset in gedachten. Het gevolg van deze verschillende manier van werken is dat software moet worden aangepast om functioneel te zijn op ARM processoren en vice versa. Dit brengt uiteraard extra werk met zich mee alsook additionele kosten. Het \textit{porten van software} zoals dit noemt, kan leiden tot meer werk dan voorzien en dit kan ook gepaard gaan met het aanschaffen van additionele test hardware. Eenmaal het porten van de software succesvol is verlopen kan het zijn dat de software nog steeds prestatie- of compatibiliteitsproblemen ondervindt. Bovenstaand probleem komt vaak voort uit het feit dat bepaalde software libraries of packages niet compatibel zijn met ARM processoren. Het porten van x86 software naar software die probleemloos functioneert op ARM toestellen is een proces dat de laatste jaren in een stroomversnelling is terechtgekomen dankzij de ontwikkeling van een nieuwe generatie aan ARM toestellen. Zowel Microsoft als Apple hebben toestellen op de markt gebracht die gebruikmaken van ARM processoren. Dit in de vorm van de bekende Snapdragon processoren die ontwikkeld worden door Qualcomm en de recente M-reeks processoren van Apple \autocite{Ford2021}.

\subsection{Doelstelling van dit onderzoek}
Het doel van deze bachelorproef is de verschillen tussen moderne ARM en x86 processoren te onderzoeken. Op basis van deze kennis is het de bedoeling om de voordelen, nadelen en energiezuinigheid van ARM processoren in kaart te brengen. Verder zal er ook onderzocht worden in welke mate het landschap van desktop besturingssystemen en software klaar is voor de coëxistentie van deze twee architecturen. Tenslotte zal een analyse uitwijzen of professionals voor hun dagelijkse taken kunnen vertrouwen op ARM computers.

%---------- Methodologie ------------------------------------------------------
\section{Methodologie}
\label{sec:methodologie}

In de eerste fase van deze studie zal de ARM architectuur onderzocht worden. Er zal ook aandacht gevestigd worden aan de gelijkenissen en verschillen tussen de ARM en x86 processor architectuur. Op basis van die kennis zullen kenmerken zoals performantie, energieverbruik, gebruiksvriendelijkheid, kostprijs en software ondersteuning bestudeerd worden. \\\\
De eerste technische proefopstelling van deze studie zal zich focussen op scenario's die per niveau zullen stijgen van complexiteit. Aan de hand van deze scenario's zal onderzocht worden welk type professional kan overschakelen naar ARM toestellen voor zijn/haar dagelijkse taken, beginnend bij kantoorsoftware en gaande tot geavanceerde software die gebruikt wordt door informatici. \\\\ 
Het tweede deel omvat een proof-of-concept applicatie die een van de grootste voordelen van de ARM processorarchitectuur zal onderzoeken, namelijk de mogelijkheid van het schrijven van software die uitvoerbaar is op verschillende types apparaten gaande van desktops tot laptops, tablets en smartphones.

%---------- Verwachte resultaten ----------------------------------------------
\section{Verwachte resultaten}
\label{sec:verwachte_resultaten}

Gezien de interesse van consumenten en professionals in ARM producten wordt er verwacht dat de meest gebruikte software vlot zal worden ondersteund op het ARM platform. Dit is het geval voor zowel Windows on ARM, MacOS en Linux. Er wordt echter wel verwacht dat Linux en MacOS een meer afgewerkte desktop ervaring zullen kunnen aanbieden bij het gebruik van een ARM processor. De oorzaak van dit fenomeen is enerzijds te verklaren door de populariteit van single board computers zoals de Raspberry Pi en de Arduino. Deze toestellen worden namelijk al enkele jaren gebruikt door professionals en hobbyisten bij projecten. Anderzijds bevinden de Mac computers zich in een transitieperiode van Intel x86 processoren naar ARM processoren. Dit gezegd zijnde wordt er verwacht dat software compatibiliteit op het ARM platform niet zal voldoen aan de compatibiliteit die men kan vinden op x86 platformen. De voornaamste tekortkomingen situeren zich op het Windows on ARM platform die zich na enkele jaren nog steeds in een testfase bevindt.

%---------- Verwachte conclusies ----------------------------------------------
\section{Verwachte conclusies}
\label{sec:verwachte_conclusies}

Met voldoende tijd en vastberadenheid kan iedereen de overstap maken naar een ARM computer voor zijn/haar dagelijkse en professionele behoeften. Dat gezegd zijnde, zal de ervaring niet altijd even gestroomlijnd zijn als op traditionele x86 hardware. De grootste problemen liggen bij de software ondersteuning, met als voornaamste probleem de bereidheid van software ontwikkelaars om hun software te porten naar ARM apparaten. Hoe meer hardware er beschikbaar is, hoe meer softwareleveranciers de noodzaak zullen inzien om hun software te porten. 



%------------------------------------------------------------------------------
% Referentielijst
%------------------------------------------------------------------------------
% TODO: de gerefereerde werken moeten in BibTeX-bestand ``voorstel.bib''
% voorkomen. Gebruik JabRef om je bibliografie bij te houden en vergeet niet
% om compatibiliteit met Biber/BibLaTeX aan te zetten (File > Switch to
% BibLaTeX mode)

\phantomsection
\printbibliography[heading=bibintoc]

\end{document}
