%%=============================================================================
%% Conclusie
%%=============================================================================

\chapter{Conclusie}
\label{ch:conclusie}

% TODO: Trek een duidelijke conclusie, in de vorm van een antwoord op de
% onderzoeksvra(a)g(en). Wat was jouw bijdrage aan het onderzoeksdomein en
% hoe biedt dit meerwaarde aan het vakgebied/doelgroep? 
% Reflecteer kritisch over het resultaat. In Engelse teksten wordt deze sectie
% ``Discussion'' genoemd. Had je deze uitkomst verwacht? Zijn er zaken die nog
% niet duidelijk zijn?
% Heeft het onderzoek geleid tot nieuwe vragen die uitnodigen tot verder 
%onderzoek?

De doelstelling van dit onderzoek is het geven van een antwoord op de onderzoeksvraag: “Is het mogelijk voor professionals en studenten om over te schakelen van de x86 architectuur naar een toestel dat beschikt over een ARM processor?”. Om een antwoord te bieden op deze vraag zijn er twee scenario’s onderzocht en is er een \textit{proof-of-concept} applicatie ontwikkeld.

Uit de resultaten van scenario 1 bleek dat Windows on ARM laptops met een Qualcomm processor en macOS toestellen die zijn uitgerust met een \textit{Apple Silicon} ARM processor kunnen gebruikt worden in een professionele kantooromgeving of voor onderwijsdoeleinden. De uitstekende beschikbaarheid van vaak gebruikte kantoorsoftware in de macOS App Store en de Windows on ARM Microsoft Store bieden een gebruikerservaring aan die niet te onderscheiden is van de ervaring die gebruikers hebben op klassieke x86-gebaseerde toestellen. Raspberry Pi OS en andere Linuxdistributies die zijn ontwikkeld voor het ARM platform hebben op dit moment nog onvoldoende software ondersteuning om tegemoet te komen aan de noden van studenten en professionals. Het volledige Microsoft Office softwarepakket is nog steeds afwezig in zowel x86 als ARM versies van Linux en het Slack communicatieplatform is enkel bruikbaar door middel van de webbrowser op ARM Linuxdistributies.

Scenario 2 heeft aangetoond dat het \textit{Apple Silicon} platform voldoende matuur is om gebruikt te worden voor sommige vormen van software ontwikkeling. De ontwikkeling van Python en Java software is mogelijk zonder dat een gebruiker problemen ondervindt die kunnen komen bij het gebruik van een verschillende processorarchitectuur. Dat gezegd zijnde is het mogelijk dat sommige verouderde of obscure aanvullende pakketten niet goed samenwerken met het nieuwe platform. Ook Raspberry Pi OS en bijgevolg het Linux on ARM platform biedt een uitstekende ervaring aan voor studenten met een achtergrond in informatica. Het enige wat het platform tegenhoudt is de geringe hoeveelheid aan performantie die deze toestellen te bieden hebben. Tenslotte heeft dit scenario aangetoond dat het Windows on ARM platform allesbehalve klaar is om gebruikt te worden in een professionele IT omgeving. Het gebrek aan softwareondersteuning en het gebrek aan engagement van externe softwareleveranciers zijn de grootste problemen die dit platform moet overwinnen indien het wil beschouwd worden als een volwaardig alternatief voor het Windows x86 platform.

Tenslotte heeft de \textit{proof-of-concept} toepassing aangetoond dat het ontwikkelen van een \textit{multiplatform} Swift applicatie op een toestel die beschikt over een \textit{Apple Silicon} processor komt met additionele voordelen. De belangrijkste daarvan is dat een ontwikkelaar toepassingen kan testen op een apparaat dat beschikt over een processor die gebruik maakt van dezelfde architectuur als deze van de beoogde eindgebruiker toestellen.

Dit onderzoek kan dienen als leidraad voor it-professionals of studenten die de overschakeling naar het ARM-platform wensen te maken. Dit platform is uitermate geschikt voor studenten en professionals die hun computer gebruiken voor alledaagse taken zoals tekstverwerking, presentaties, digitale rekenbladen, browsen en online communicatie. Studenten en professionals met een informatica achtergrond kunnen de overstap maken naar een \textit{Apple Silicon} toestel indien men bereid is om grondig onderzoek te doen omtrent de software ondersteuning van de applicaties die nodig zijn voor hun alledaagse taken. Raspberry Pi toestellen zijn uiterst geschikt voor projecten en testopstellingen die geen nood hebben aan een krachtige processor. Het gebrek aan software ondersteuning op het Windows on ARM platform zorgt ervoor dat deze toestellen op dit moment nog niet geschikt zijn voor it-professionals en studenten. Alleen wanneer de software ondersteuning verbetert tot op het niveau van het Windows x86 platform, zullen deze professionals en studenten in staat zijn om gebruik te maken van een Windows on ARM toestel. 

Het desktop ARM platform is nog relatief nieuw. Het niveau van software ondersteuning is de afgelopen jaren exponentieel toegenomen en het is indrukwekkend hoe sommige professionals reeds gebruik kunnen maken van deze hardware voor hun alledaagse activiteiten. Om het huidige tempo van hardware- en software-innovatie op het ARM platform bij te houden, is het zinvol om deze studie te herhalen op jaarlijkse basis. Dit kan eventueel onderzocht worden in een verder onderzoek.