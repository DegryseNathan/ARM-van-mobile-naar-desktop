%%=============================================================================
%% Samenvatting
%%=============================================================================

% TODO: De "abstract" of samenvatting is een kernachtige (~ 1 blz. voor een
% thesis) synthese van het document.
%
% Deze aspecten moeten zeker aan bod komen:
% - Context: waarom is dit werk belangrijk?
% - Nood: waarom moest dit onderzocht worden?
% - Taak: wat heb je precies gedaan?
% - Object: wat staat in dit document geschreven?
% - Resultaat: wat was het resultaat?
% - Conclusie: wat is/zijn de belangrijkste conclusie(s)?
% - Perspectief: blijven er nog vragen open die in de toekomst nog kunnen
%    onderzocht worden? Wat is een mogelijk vervolg voor jouw onderzoek?
%
% LET OP! Een samenvatting is GEEN voorwoord!

%%---------- Nederlandse samenvatting -----------------------------------------
%
% TODO: Als je je bachelorproef in het Engels schrijft, moet je eerst een
% Nederlandse samenvatting invoegen. Haal daarvoor onderstaande code uit
% commentaar.
% Wie zijn bachelorproef in het Nederlands schrijft, kan dit negeren, de inhoud
% wordt niet in het document ingevoegd.

\IfLanguageName{english}{%
\selectlanguage{dutch}
\chapter*{Samenvatting}
\lipsum[1-4]
\selectlanguage{english}
}{}

%%---------- Samenvatting -----------------------------------------------------
% De samenvatting in de hoofdtaal van het document

\chapter*{\IfLanguageName{dutch}{Samenvatting}{Abstract}}

Deze studie is gericht op het bestuderen van de ARM processorarchitectuur die momenteel voornamelijk wordt gebruikt in toestellen zoals smartphones en tablets. Naast het gebruik in apparaten met een focus op mobiliteit, zijn ARM processoren een integraal onderdeel van \textit{Internet of Things} (IoT) toestellen en \textit{single-board} computers zoals de populaire Raspberry Pi, waar ze terug te vinden zijn in een \textit{System-on-a-Chip} (SoC) configuratie. Onderzoek naar de architectuur van ARM processoren heeft uitgewezen dat deze komen met enkele significante voor- en nadelen, zijnde toonaangevende energiezuinigheid en de nood aan speciaal ontworpen software. Na research omtrent de prestatie per watt eigenschappen van ARM processoren kan er geconcludeerd worden dat deze architectuur gepaard gaat met significante reducties op het gebied van energieverbruik en warmte uitstoot. Deze eigenschappen zijn terug te vinden in toestellen die gericht zijn op consumenten, alsook bij serverapparatuur in datacenters waar ARM processoren worden ontworpen voor intern gebruik. Tenslotte heeft een marktonderzoek inzicht verschaft in welke ARM computers er beschikbaar zijn voor consumenten en professionals. Deze komen in de vorm van de \textit{Apple Silicon} toestellen en de Windows on ARM laptops die gebruikmaken van processoren die  ontwikkeld zijn door Qualcomm. Scenario’s en een \textit{proof-of-concept} applicatie hebben vastgesteld dat computers die zijn uitgerust met een ARM processor, een valabel alternatief vormen voor x86 computers voor algemeen gebruik en voor geavanceerde gebruikers zoals professionele informatici of studenten met een focus op informatica.
