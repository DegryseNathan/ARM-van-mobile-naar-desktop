%%=============================================================================
%% Voorwoord
%%=============================================================================

\chapter*{\IfLanguageName{dutch}{Woord vooraf}{Preface}}
\label{ch:voorwoord}

%% TODO:
%% Het voorwoord is het enige deel van de bachelorproef waar je vanuit je
%% eigen standpunt (``ik-vorm'') mag schrijven. Je kan hier bv. motiveren
%% waarom jij het onderwerp wil bespreken.
%% Vergeet ook niet te bedanken wie je geholpen/gesteund/... heeft

Deze bachelorproef vertegenwoordigt de synthese van mijn opleiding Toegepaste Informatica met een specialisatie in systeem- en netwerkbeheer. Tijdens deze opleiding had ik het genoegen om vele gebieden in de computerwetenschappen te verkennen. Het is mijn passie voor hardware, software en besturingssystemen die me ertoe gebracht heeft om de fascinerende wereld van de ARM processor architectuur te ontdekken.

Deze thesis zou niet mogelijk zijn geweest zonder de hulp, feedback en input van verscheidene personen. In de volgende secties wil ik deze personen bedanken voor al hun inspanningen. 

In de eerste plaats wil ik uiteraard mijn co-promotor Mathieu Audenaert bedanken voor de hulp tijdens dit onderzoek, de feedback en de gespendeerde tijd. Het is geen overschatting om te stellen dat Mr. Audenaert mede verantwoordelijk is voor mijn interesse en passie voor computerwetenschappen en computerhardware.

Ook wil ik mijn promotor, Sharon Van Hove uitdrukkelijk bedanken voor al haar input, suggesties, feedback en ondersteuning. Ik kon haar steeds bereiken om vragen te stellen over de inhoud van deze thesis, over het taalgebruik en over ideeën die dit project zouden kunnen verbeteren.

Ik wil ook mijn naaste vrienden bedanken die de tijd hebben genomen om deze scriptie te lezen. Het is dankzij hun inbreng en kritische doch correcte feedback dat ik het kwaliteitsniveau heb kunnen halen dat ik voor ogen had.

Tenslotte wens ik ook een dankwoord te richten aan mijn ouders die mij gedurende mijn traject als student altijd hebben ondersteund en geïnspireerd. 

Ik wens u veel leesplezier met deze bachelorproef en ik hoop dat deze paper nuttig zal blijken voor uw onderzoek.
