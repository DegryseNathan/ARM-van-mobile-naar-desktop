%%=============================================================================
%% Methodologie
%%=============================================================================

\chapter{\IfLanguageName{dutch}{Methodologie}{Methodology}}
\label{ch:methodologie}

%% TODO: Hoe ben je te werk gegaan? Verdeel je onderzoek in grote fasen, en
%% licht in elke fase toe welke stappen je gevolgd hebt. Verantwoord waarom je
%% op deze manier te werk gegaan bent. Je moet kunnen aantonen dat je de best
%% mogelijke manier toegepast hebt om een antwoord te vinden op de
%% onderzoeksvraag.

Technologische details zoals performantie en energiezuinigheid worden quasi irrelevant wanneer professionals en studenten niet in staat zijn om de softwaretools te gebruiken die ze nodig hebben voor hun dagelijkse taken. Opdat professionals en studenten bereid zouden zijn om over te stappen naar een laptop of desktop die gebruik maakt van de ARM architectuur, moeten de gebruiksvriendelijkheid en software ondersteuning gelijkwaardig zijn aan die van het x86-platform. Het is niet ongewoon dat ontwikkelaars problemen en onregelmatigheden ondervinden bij het gebruik van bepaalde softwarepakketten, IDE’s of enigszins verouderde programmeertalen. Gewone computergebruikers zijn echter niet vertrouwd met soortgelijke problemen die hun werkstroom kunnen onderbreken. Vandaar dat het essentieel is voor ARM computers om een stabiele en verfijnde gebruikerservaring te voorzien voor gewone computergebruikers. ICT professionals zijn vertrouwd met het aanpassen en sleutelen van software zodat deze een ervaring biedt die perfect is afgestemd op hun eigen unieke gebruikssituatie. Dit gezegd zijnde is het uiteraard essentieel dat hun taken zonder al te veel problemen en onderbrekingen worden uitgevoerd en voltooid.

Dit onderdeel van de studie is gericht op het onderzoeken van de software compatibiliteit en het gebruiksgemak van ARM-gebaseerde desktop-besturingssystemen. Scenario 1 is gericht op alledaags gebruik zoals browsen en kantoorsoftware. In scenario 2 komen de geavanceerde ICT toepassingen aan bod die deel uitmaken van het dagelijkse softwarepakket van een student toegepaste informatica of een ICT professional.

Het is uiteraard onmogelijk om elke vorm van computergebruik te onderzoeken, vandaar dat elke gekozen toepassing bij de scenario’s gepaard zal gaan met een verklaring omtrent de relevantie ervan.

Deze praktische opstellingen zullen uitgevoerd worden aan de hand van twee toestellen. Het eerste apparaat is een Raspberry Pi 4 die beschikt over een \textit{quad core} Cortex-A72 (ARM v8) 64-bit SoC en 4GB LPDDR4-3200 SDRAM \autocite{Pi2021}. Dit toestel maakt gebruik van Raspberry Pi OS, voorheen bekend als \textit{Raspbian}. Deze Linuxdistributie maakt deel uit van de \textit{Debian} familie en is grotendeels gebaseerd op de populaire Ubuntu \textit{distro}. Raspberry Pi OS is echter sterk geoptimaliseerd en gespecialiseerd voor de familie van Raspberry Pi toestellen die gebruikmaken van ARM processoren. Het voornaamste verschil met andere Linuxdistributies is de relatief eenvoudige LXDE desktopomgeving die is aangepast om minder resources te verbruiken \autocite{Pi2011}.

Het tweede apparaat dat deel uitmaakt van deze studie is een 13‑inch MacBook Pro die is uitgerust met een M1 \textit{Apple Silicon} processor. Dit toestel zal gebruikt worden om de gebruikerservaring en de software ondersteuning te testen op het macOS (ARM) platform. Bij het ontwikkelen van deze nieuwe processorarchitectuur heeft Apple ernaar gestreefd om een desktopervaring te bieden die evenwaardig is aan of beter dan het huidige macOS x86-64 platform \autocite{Apple2020}. Dit onderzoek zal deze gedurfde marketing beweringen uiteraard in vraag stellen en testen.

Om het Windows 11 on ARM besturingssysteem te testen zal dit onderzoek gebruikmaken van een virtuele opstelling door middel van Parallels Desktop 17. Deze versie van de bekende virtualisatiesoftware ondersteunt de \textit{Apple Silicon} architectuur en voorziet een goede integratie tussen het host- en gevirtualiseerde systeem. Deze werkwijze gaat echter wel gepaard met een bepaald percentage aan prestatieverlies en enkele eigenaardigheden, echter in het kader van dit onderzoek zal deze virtuele benadering volstaan.

\begin{figure}[h]
	\centering
	\includegraphics[width=100mm, scale=0.5]{img/specificaties_winARM.png}
	\caption{Specificaties van het gevirtualiseerde Windows 11 on ARM systeem}
	\end{figure}

\newpage
\section{Scenario 1: alledaags gebruik}
Dit eerste scenario omvat software die vrijwel ondenkbaar is bij hedendaagse kantoorjobs en bachelor/master opleidingen. Een studie gevoerd door \textcite{Braganza2022} bracht aan het licht dat browsers gebaseerd op de Chromium \textit{engine}, Microsoft Office en Slack veruit de meest gebruikte applicaties zijn door professionals met een kantoorjob. Indien laptops en desktops met ARM processoren willen doorbreken bij het grote publiek, is het essentieel dat deze programma’s een gebruikerservaring aanbieden die niet te onderscheiden is van deze op x86 toestellen.

\subsection{Raspberry Pi OS}
Microsoft Office is tot op de dag van vandaag nog steeds veruit de meest populaire office suite. Deze populariteit kan verklaard worden door de dominantie in de onderwijs- en businessmarkt. Ondanks deze populariteit is Microsoft Office niet beschikbaar op Linux gebaseerde besturingssystemen. In dit scenario is de keuze tussen een x86 of een ARM systeem dus irrelevant. Desondanks dit opvallend tekort zijn Linux computers niet afgeschreven voor kantoorgebruik. LibreOffice is sinds jaar en dag een uitstekende optie voor gebruikers die niet wensen te betalen voor een Office 365 abonnement of een eenmalige aankoop van Microsoft Office. LibreOffice valt onder de noemer van de opensourcesoftware en is bijgevolg gratis voor de eindgebruiker. Sommigen zullen de grafische gebruikersomgeving wat verouderd en rudimentair vinden, maar verder biedt dit softwarepakket een gebruikerservaring die evenwaardig is met andere office software. Indien een gebruiker zou overschakelen naar LibreOffice, moet deze aanvankelijk bereid zijn om sommige taken opnieuw te leren, maar na een korte transitieperiode zou hij/zij geen verdere tekortkomingen of problemen mogen ondervinden. Dit gezegd zijnde, is niet elk bedrijf bereid om Microsoft Office aan de kant te schuiven voor een ander product louter op basis van het feit dat het merendeel van de medewerkers vertrouwd zijn met de Microsoft variant. Indien een gebruiker toch wenst gebruik te maken van Microsoft Office, dan is dit mogelijk via de browser. Deze variant biedt echter niet dezelfde functionaliteit als de desktop variant, maar voor alledaags gebruik zou deze moeten volstaan.

\begin{figure}[!h]
	\centering
	\includegraphics[width=110mm, scale=0.7]{img/office_pi.png}
	\caption{LibreOffice en Office 365 op de Raspberry Pi 4}
\end{figure}

\newpage
Slack is het communicatiemiddel bij uitstek in de bedrijfswereld en is vandaar uiteraard essentieel bij een moderne kantoorjob. Deze tool wordt integraal ondersteund op x86 Linux besturingssystemen, maar is echter opvallend afwezig op Raspberry Pi OS. Een mogelijke oplossing voor dit tekort is het gebruiken van Slack in de Chromium browser, maar dit is voor velen geen kwalitatief alternatief.

\begin{figure}[!h]
	\centering
	\includegraphics[width=110mm, scale=0.7]{img/slack_pi.png}
	\caption{Slack op de Raspberry Pi 4}
\end{figure}

Chromium is de standaardbrowser op het Raspberry Pi OS besturingssysteem. Dit is een open source variant van de Google Chrome webbrowser die gericht is op eenvoud, stabiliteit en performantie \autocite{ChromiumProject2009}. Een van de grootste nadelen die komt bij het gebruik van Chromium is dat het geen integratie met Google services ondersteunt. Een gebruiker kan bijvoorbeeld een account aanmaken, maar dit is enkel een lokaal account. Het is bijgevolg niet mogelijk om in te loggen op een werk- of privé Google-account waar men bladwijzers, data of andere instellingen kan importeren. Verder zijn de populaire Google Chrome en Microsoft Edge browsers niet ondersteund op Linux besturingssystemen die uitgevoerd worden op een ARM processor. Mozilla’s Firefox webbrowser is het enige volwaardige alternatief voor Chromium.

\subsection{macOS Monterey Apple Silicon}
De situatie verbetert aanzienlijk wanneer er wordt overgeschakeld naar macOS Monterey op een \textit{Apple Silicon} toestel. Zoals eerder werd aangehaald, verstrekte Apple enkele maanden voor de officiële release van hun nieuwe architectuur een \textit{Developer Transition Kit} (DTK). Dit toestel gaf ontwikkelaars de mogelijkheid om hun bestaande software of hun nieuwe projecten te hercompileren zodat ze een \textit{native} versie konden aanbieden \autocite{Apple2020}. Deze aanpak lijkt grotendeels succesvol te zijn geweest aangezien alle te testen applicaties in dit scenario \textit{native} verkrijgbaar zijn en een gebruikerservaring aanbieden die niet te onderscheiden is van macOS Monterey die draait op een x86 systeem.

Slack en alle Microsoft Office applicaties afgezien van Microsoft Teams, zijn vrij te verkrijgen in de App Store. Microsoft Teams dient gedownload te worden via de browser, dit is echter ook het geval bij de x86 variant van macOS Monterey. Een eindgebruiker heeft als het ware geen idee dat zijn/haar applicaties actief zijn op een systeem dat gebruik maakt van een processorarchitectuur die verschillend is van hetgeen waar hij/zij tot dan toe vertrouwd mee was. Naast het betaalde Microsoft Office pakket stelt Apple hun iWork suite gratis beschikbaar voor iedereen die een \textit{Apple ID} heeft. Dit softwarepakket is geen volwaardig alternatief voor Microsoft Office, maar ze biedt voldoende mogelijkheden voor gewone eindgebruikers en studenten.

\begin{figure}[!h]
	\centering
	\includegraphics[width=110mm, scale=0.7]{img/office_macOSM1.png}
	\caption{Pages en Microsoft Word op het Apple Silicon platform}
\end{figure}

De enige manier waarop een eindgebruiker te weten kan komen of een applicatie gecompileerd is voor de \textit{Apple Silicon} architectuur of gebruikmaakt van de \textit{Rosetta 2 translation layer} is de activiteitenweergave applicatie. Deze biedt een handig overzicht van de gebruikte processorarchitectuur in de \textit{kind} kolom. In het onderstaande voorbeeld is er te zien dat Microsoft OneDrive en WhatsApp nog niet \textit{native} ondersteund zijn.

\begin{figure}[!h]
	\centering
	\includegraphics[width=110mm, scale=0.7]{img/activitymonitor_macOSM1.png}
	\caption{Een overzicht van de activiteitenweergave applicatie}
\end{figure}

Wat webbrowsers betreft is er geen tekort aan opties op macOS. Google Chrome, Microsoft Edge, Mozilla Firefox, Brave enzovoort worden allemaal \textit{native} ondersteund op het \textit{Apple Silicon} platform. Wanneer een gebruiker een browser naar keuze wenst te downloaden, krijgt deze een dialoogvenster te zien waar hij/zij kan kiezen tussen de Intel x86 variant of de versie die gericht is op het \textit{Apple Silicon} platform.

\begin{figure}[!h]
	\centering
	\includegraphics[width=110mm, scale=0.7]{img/browserversie_macOSM1.png}
	\caption{De Microsoft Edge download pagina \autocite{Microsoft2022}}
\end{figure}

\subsection{Windows 11 on ARM}
Net als op macOS voor de \textit{Apple Silicon} architectuur, zijn de meeste \textit{native} Windows 11 on ARM applicaties verkrijgbaar via de appstore die gericht is op het eigen platform, in dit geval is dit de Microsoft Store die voorheen bekend stond als de Windows Store. Slack heeft een \textit{native} versie die compatibel is met Windows 10 en Windows 11 on ARM en deze bevat alle toepassingen die aanwezig zijn op het Slack communicatieplatform.

Microsoft Office heeft ook een \textit{native} Windows on ARM versie, deze is echter vreemd genoeg niet terug te vinden in de Microsoft Store. Een gebruiker dient deze te downloaden via de Office 365 portaalsite. Verder krijgt een gebruiker geen indicatie van welke versies van Microsoft Office er beschikbaar zijn om te downloaden. Enerzijds beperkt deze aanpak mogelijke verwarring tot een minimum, maar anderzijds is het belangrijk om te weten of de \textit{native} versie beschikbaar is, aangezien deze stabieler is en zorgt voor betere prestaties. Net als op het macOS platform kan een gebruiker op Windows 11 on ARM de architectuur van een applicatie raadplegen. Dit wordt gedaan door te navigeren naar de taakbeheer applicatie en te klikken op de “details” kolom \autocite{Venkat2020}.

\begin{figure}[!h]
	\centering
	\includegraphics[width=\linewidth]{img/office_winARM.png}
	\caption{Microsoft Office op het Windows on ARM platform}
\end{figure}

Op het gebied van browserondersteuning stelt het Windows on ARM platform enigszins teleur. Microsoft Edge is zoals altijd de standaardbrowser voor Windows toestellen, deze voorziet alle functionaliteit die Microsoft Edge biedt op een x86 systeem. De problemen komen echter opdagen wanneer een gebruiker een andere browser wenst te gebruiken. Google biedt tot op de dag van vandaag geen ARM variant aan van hun marktleidende Chrome webbrowser terwijl Mozilla Firefox en de open source Chromium browser wel vlot verkrijgbaar zijn voor het Windows on ARM platform. Microsoft treft geen blaam voor deze tekortkoming, het voorspelt echter niet veel goeds voor de platformondersteuning wanneer een technologiegigant als Google zijn software niet ter beschikking stelt.

\begin{figure}[!h]
	\centering
	\includegraphics[width=110mm, scale=0.7]{img/firefox_winARM.png}
	\caption{Mozilla Firefox op het Windows on ARM platform \autocite{Mozilla2022}}
\end{figure}

\section{Scenario 2: professioneel IT-gebruik}
In dit tweede scenario komen de programma’s aan bod die deel uitmaken van het softwarepakket van een programmeur of een student toegepaste informatica of aanverwante opleiding. De programmeertalen die getest zullen worden in dit onderzoek zijn gekozen op basis van de populariteitsindex van de jaarlijkse \textit{Stack Overflow Developer Survey} en het feit of ze al dan niet aan bod komen in een opleidingsonderdeel van de professionele bachelor toegepaste informatica.

Python bevindt zich al enkele jaren in de top 10 van de meest geliefde en gebruikte programmeertalen en deze wordt eveneens gebruikt in de opleiding bij het opleidingsonderdeel \textit{Classical Computer Science Algorithms}. Verder zal de compatibiliteit van de Java programmeertaal ook onderzocht worden aangezien deze nog steeds populair is bij zowel softwarebedrijven en onderwijsinstellingen \autocite{StackOverflow2020}. Naast software ontwikkeling zullen ook versiebeheer, virtualisatie, containervirtualisatie en de computerterminal op de proef gesteld worden. Deze testopstellingen zullen moeten uitwijzen of de gebruikerservaring en de software ondersteuning op ARM besturingssystemen gelijkwaardig zijn aan deze op populaire en vertrouwde x86 besturingssystemen.

\subsection{Versiebeheer met Git}
Als informaticus is het quasi onmogelijk om werk gedaan te krijgen of samen te werken met collega’s zonder het gebruik van een versiebeheersysteem. Git is veruit de meest gebruikte en aanvaarde standaard als het komt op het gebied van versiebeheer, vandaar dat dit onderzoek zich hierop zal richten \autocite{Banerjee2022}.

Raspberry Pi- en macOS maken het de gebruiker gemakkelijk aangezien deze twee besturingssystemen standaard Git functionaliteit hebben ingebouwd terwijl dit op het Windows on ARM platform dient geïnstalleerd te worden door de eindgebruiker. De Git \textit{client} is echter niet \textit{native} gecompileerd voor het Windows on ARM platform. Om het gebrek aan \textit{native} software te verhelpen, voorziet Windows on ARM emulatie van sommige x86 programma’s. Het is echter belangrijk om op te merken dat deze aanpak niet altijd werkt. 

Alle bovengenoemde platformen voeren de grafische en console variant van de Git \textit{client} uit zonder enige vorm van problemen en de gebruikerservaring is niet te onderscheiden van deze op x86 platformen.

\subsection{Programmeren in Python}
De Python programmeertaal is sinds zijn ontstaan geliefd bij zowel professionele ontwikkelaars en studenten omwille van de beginnersvriendelijke syntaxis en de geringe hoeveelheid aan performantie die een computer nodig heeft om met deze programmeertaal te werken \autocite{StackOverflow2020}. Dit betekent echter niet dat een ontwikkelaar niet in staat is om complexe algoritmes to coderen in Python. Om te onderzoeken of een ontwikkelaar of een student een ARM computer kan gebruiken om in Python te programmeren, zal deze sectie de beschikbaarheid van de programmeertaal, de beschikbaarheid van een geschikte \textit{integrated development environment} (IDE) en de prestaties van een veeleisende Python \textit{benchmark} evalueren.

De Python programmeertaal is een softwarepakket dat standaard geïnstalleerd is op de meeste Linuxdistributies. Dit is ook het geval bij het op ARM gebaseerde Raspberry Pi OS waar zowel versie 2 als 3 beschikbaar zijn voor de gebruiker. Verder stelt Raspberry Pi OS enkele eenvoudige Python IDE’s ter beschikking als onderdeel van de voorgeïnstalleerde software. Deze komen in de vorm van Geany en Thonny, dit zijn beide Python programmeeromgevingen die gericht zijn op de markt van beginnende programmeurs of kinderen en tieners die hun eerste lijnen code willen schrijven \autocite{Fromaget2021}. Verder kan een gebruiker de populaire Visual Studio Code broncode-editor downloaden via het \textit{Advanced Package Tool} (APT) pakketbeheerprogramma die terug te vinden is op alle Linuxdistributies die gebaseerd zijn op Debian.

Windows 11 on ARM biedt eveneens een uitstekende gebruikerservaring aan op het gebied van de installatie van de Python software en de Visual Studio Code editor. Deze zijn immers allebei terug te vinden in de Microsoft Store. Python programmeurs op het macOS besturingssysteem dienen Python en VS Code te downloaden via een browser naar keuze of de Homebrew \textit{package manager}.

Om de prestaties van de drie verschillende ARM systemen te evalueren, is op alle toestellen een veeleisende Python \textit{benchmark} uitgevoerd. Deze komt in de vorm van het Mandelbrot algoritme die gebaseerd is op de wiskundige verzameling genaamd naar de Pools-Franse wiskundige Benoit Mandelbrot. De Mandelbrot \textit{benchmark} genereert fractalen, dit zijn zelfgelijkende meetkundige figuren. Fractalen zijn als het ware complexe wiskundige figuren die zijn opgebouwd uit figuren die deels gelijkvormig zijn met de figuur zelf. Het praktisch nut van wiskundige fractalen wordt gebruikt in de chaostheorie, een wiskundige discipline die zich focust op de studie van het gedrag van niet-lineaire dynamische systemen \autocite{Mandelbrot1980}.

Het algoritme dat wordt gebruikt om Mandelbrot fractalen te berekenen en genereren stelt de mogelijkheden van processoren zwaar op de proef. In de onderstaande afbeelding is de uitvoer en de uitvoeringstijd van het Mandelbrot algoritme te zien met een stapgrootte van 16000 fractalen. Dit houdt in dat het algoritme 16000 fractalen moet berekenen en genereren. Het spreekt voor zich dat de \textit{Apple Silicon} M1 de kortste uitvoeringstijd heeft aangezien deze processor een hogere kloksnelheid heeft en meer processorkernen bezit dan de andere toestellen in deze studie. De gevirtualiseerde Windows on ARM machine komt op de tweede plaats met een uitvoeringstijd die vier keer langer is. Het is echter belangrijk om te kaderen dat deze virtuele machine slechts de helft van de processorkernen bezit waartoe het host M1 toestel toegang tot heeft. De uitvoeringstijd van het Mandelbrot algoritme op de Raspberry Pi 4 bedraagt 470 seconden, dit is zowat acht keer langer dan het snelste systeem dat in deze test werd gebruikt. De indrukwekkendheid van de Mandelbrot test op het Raspberry Pi 4 systeem wordt pas duidelijk wanneer de prijs ervan wordt vergeleken met de andere toestellen die werden gebruikt voor deze research. De Pi 4 heeft namelijk een prijskaartje dat elf keer kleiner is dan dit van het M1 toestel.

Aan de hand van de resultaten van de Mandelbrot performantietest en aan de uitstekende beschikbaarheid van Python ontwikkelsoftware is het plausibel om te stellen dat moderne systemen die zijn uitgerust met ARM processoren geschikt zijn voor Python ontwikkeling. De Raspberry Pi 4 is in staat om uitstekende performantie te leveren voor een geringe kost en laat op die manier zien waarom hij door scholen en universiteiten over de hele wereld wordt gebruikt om studenten te leren programmeren.

\begin{figure}[!h]
	\centering
	\includegraphics[width=\linewidth]{img/mandelbrot_benchmark.png}
	\caption{Een overzicht van de Mandelbrot benchmark}
\end{figure}

\subsection{Programmeren in Java}
Java is nog steeds een populaire programmeertaal in het jaar 2022. Naast Java staat ook Kotlin, die grotendeels gebaseerd is op Java, bovenaan de ranglijsten. De populariteit van Java en Kotlin is grotendeels te danken aan het Android besturingssysteem die terug te vinden is in toestellen gaande van smartphones tot slimme koelkasten. Het spreekt dus voor zich dat een groot aandeel van de ontwikkelaars en softwarebedrijven Java ontwikkeling zien als een van hun belangrijkste prioriteiten \autocite{StackOverflow2020}. 

Om aan Java ontwikkeling te kunnen doen, heeft een ontwikkelaar nood aan de \textit{Java Runtime Environment}, \textit{Java Development Kit} en een\textit{ integrated development environment} (IDE). Op het Raspberry Pi- en macOS platform voor ARM toestellen, zijn deze te verkrijgen via de website van Oracle of via de APT en Homebrew pakketbeheerprogramma's. IntelliJ, Eclipse, NetBeans en BlueJ zijn allen gerespecteerde en vaak gebruikte IDE’s om aan Java ontwikkeling te doen in een professionele of educatieve omgeving. In het kader van dit onderzoek werd Eclipse gekozen als de geprefereerde IDE aangezien deze wordt gebruikt in verschillende opleidingsonderdelen. De Eclipse Java IDE is kosteloos te verkrijgen via de webpagina van de Eclipse Foundation \autocite{EclipseFoundation2022}.

Om de gebruikerservaring van Java ontwikkeling met Eclipse op de ARM versies van Linux en macOS te testen is er geopteerd geweest om enkele opdrachten te herprogrammeren gaande van \textit{Object-oriented software development 1} tot \textit{Advanced Software development 1}. Deze zijn allen zonder problemen uitgevoerd op het Raspberry Pi 4 systeem en de met M1 \textit{Apple Silicon} uitgeruste Macbook.

Op het Windows 11 on ARM toestel verliep de ervaring echter minder vlot. De enige manier waarop de \textit{Java Development Kit} te verkrijgen is, is via een blogpost op het 
Microsoft Developers forum \autocite{Borges2020}. Dit is allesbehalve een voordehandliggende en gebruikersvriendelijke oplossing. Het gebrek aan adequate ondersteuning wordt alleen maar duidelijker wanneer men de Eclipse editor probeert te verkrijgen. Deze is namelijk beschikbaar voor elk platform behalve Windows on ARM. Tot overmaat van ramp leverde de zoektocht naar een volwaardig alternatief voor Eclipse geen resultaten op. De enige manier om aan Java ontwikkeling te doen op het Windows on ARM platform, is door middel van een \textit{text editor} zoals Vim, Emacs of Nano. Een \textit{text editor} geeft in tegenstelling tot een \textit{code editor} geen feedback of code-suggesties. Deze aanpak is vaak voordelig voor ervaren programmeurs die reeds een meesterschap hebben bereikt in een bepaalde programmeertaal, maar dit is echter niet aan te raden voor beginners en studenten.

\begin{figure}[!h]
	\centering
	\includegraphics[width=\linewidth]{img/java_development.png}
	\caption{Een overzicht Java ontwikkeling op de Raspberry Pi 4 en Apple Silicon \autocite{EclipseFoundation2022}}
\end{figure}

\subsection{Virtualisatiesoftware}
Virtualisatiesoftware is essentieel voor elke ICT professional en student. Het stelt de gebruiker in staat om een virtuele instantie te creëren van een besturingssysteem naar keuze. Een praktische toepassing is bijvoorbeeld studenten die Linux wensen te installeren en gebruiken zonder zich er voor een langere tijd aan te verbinden of zonder de nood aan een extra toestel. Verder kan virtualisatie de flexibiliteit en schaalbaarheid van een onderneming vergroten en aanzienlijke kostenbesparingen opleveren. Het kan tenslotte ook leiden tot betere prestaties en een betere beschikbaarheid van middelen, wat op termijn leidt tot eenvoudiger IT management die gepaard gaat met lagere beheer- en eigendomskosten \autocite{VMware2016}.

Gezien de geringe hoeveelheid performantie waartoe de Raspberry PI 4 toegang tot heeft, zou het onverstandig zijn om middelen toe te wijzen aan een klassieke hypervisor. Dit betekent echter niet dat de Pi niet in staat is om gevirtualiseerde Docker containers te draaien die een minimum aan impact hebben op het algemene gebruiksgemak. Het gebruik van Docker op Raspberry Pi OS komt later dit hoofdstuk nog aan bod.

VirtualBox is een vaak gebruikte virtualisatie tool door professionele informatici en studenten. Het stelt gebruikers in staat om een groot aantal aan besturingssystemen te virtualiseren op een x86 instantie van Linux, macOS of Windows \autocite{VirtualBox2003}. Deze kerndoelstelling is dan ook terug te vinden op de voorpagina van hun website. Dit impliceert dat de makers van VirtualBox, Oracle, zich gefocust hebben op de markt van x86 emulatie en dus geen verdere intenties hebben om zich te richten op de markt van ARM virtualisatie.

\begin{figure}[!h]
	\centering
	\includegraphics[width=\linewidth]{img/virtualbox.png}
	\caption{De missie van VirtualBox \autocite{VirtualBox2003}}
\end{figure}

Parallels Desktop is software die gebruikt wordt voor hardwarevirtualisatie op het macOS en Chrome OS besturingssysteem. De Parallels hypervisor komt met geavanceerde functies en een uitstekende integratie met het host systeem. Zo zijn bijvoorbeeld alle bestanden op het hostsysteem beschikbaar op het gevirtualiseerde systeem \autocite{Parallels2009}. Sinds versie 16.5 biedt Parallels ondersteuning voor \textit{Apple Silicon} processors en is het bijgevolg mogelijk om ARM besturingssystemen te virtualiseren \autocite{Vogel2021}. Hiertoe behoren Windows 10 en 11 on ARM en alle Linuxdistributies die een ARM versie beschikbaar stellen. 

Parallels is een eenvoudig te gebruiken hypervisor die voornamelijk gericht is op het bieden van een uitstekende ervaring aan beginnende gebruikers die niet noodzakelijk informatici zijn \autocite{Parallels2009}. Dit wordt weerspiegeld in het venster waar een gebruiker een nieuwe virtuele machine kan creëren. Daar kan een gebruiker kiezen tussen enkele van de meest voorkomende besturingssystemen. Dit betekent echter niet dat Parallels geen opties bevat voor geavanceerde gebruikers. Het is nog steeds mogelijk om een virtuele machine aan te maken op basis van een ISO-bestand dat de gebruiker eerder heeft gedownload.

De Parallels Desktop hypervisor biedt een uitstekende gebruikerservaring voor zowel beginnende als gevorderde gebruikers. Hier hangt echter een prijskaartje aan. Een licentie kost tussen de tachtig en honderd euro, afhankelijk van de versie. Verder is het ook essentieel om te vermelden dat het enkel mogelijk is om de ARM versies van besturingssystemen te virtualiseren. Als gebruiker ben je dus afhankelijk van het niveau van softwareondersteuning dat de maker van een besturingssysteem biedt.

\begin{figure}[!h]
	\centering
	\includegraphics[width=\linewidth]{img/parallels.jpg}
	\caption{Een overzicht van Parallels Desktop 17}
\end{figure}

\textit{Windows Subsystem for Linux} (WSL) is een compatibiliteitslaag die beschikbaar is op Windows besturingssystemen. Het biedt een gevirtualiseerde Linux-omgeving aan waarin de gebruiker Linux-tools en -programma’s kan gebruiken. Het is belangrijk om op te merken dat deze zich voornamelijk situeren in de computerterminal. Het installatieproces van WSL is identiek op x86 en ARM instanties van Windows, maar een systeem moet gebruikmaken van Windows 10 versie 2004 en hoger of Windows 11 \autocite{Microsoft2021}. De installatie van een Linuxdistributie naar keuze gebeurt via de Microsoft Store waar de meest gebruikte opties beschikbaar zijn. 

\begin{figure}[!h]
	\centering
	\includegraphics[width=\linewidth]{img/debian.png}
	\caption{Het Debian Linux besturingssysteem in de Microsoft Store}
\end{figure}

De integratie van WSL in Windows on ARM geeft gebruikers een basis Linux-omgeving die kan gebruikt worden voor simpele taken, maar het is geen volwaardige vervanging voor een hypervisor zoals VirtualBox, Parallels of VMware.

\subsection{Docker containervirtualisatie}
Docker is een platform die gebruikt wordt voor het ontwikkelen, distribueren en uitvoeren van applicaties, verder stelt het de gebruiker in staat om zijn/haar applicaties te scheiden van de eigen infrastructuur. Het kan gebruikt worden om code te delen, het pushen van een applicatie naar een testomgeving, het zoeken naar software bugs enzovoort. Docker heeft een minimale impact op de performantie van het host systeem en het biedt een kosteneffectief alternatief voor virtuele machines die gebaseerd zijn op hypervisors. Docker maakt gebruik van \textit{images}, dit zijn \textit{read-only} sjablonen met instructies om een \textit{container} te maken. Deze modulaire aanpak maakt het gemakkelijk om een bestaande \textit{image} aan te passen naargelang de eigen noden. Een gebruiker kan een \textit{image} specificeren aan de hand van een Dockerfile die is opgebouwd uit een eenvoudige syntaxis. Wanneer een gebruiker wijzigingen aanbrengt in deze file en het \textit{image} heropbouwt, worden enkel de lagen veranderd die zijn heropgebouwd. Deze eigenschap zorgt voor de uitstekende performantie en snelheid van \textit{images}, in vergelijking met andere virtualisatietechnieken \autocite{Docker2020}.

Docker kan gemakkelijk geïnstalleerd worden op de ARM edities van Linux en macOS door middel van de APT en Homebrew pakketbeheerprogramma’s. Om te verifiëren of de installatie van Docker succesvol is, volstaat het om de “hello-world” \textit{container} uit te voeren \autocite{Andrews2019}. Deze zou de volgende tekst moeten opleveren:

\begin{lstlisting}
Hello from Docker!
This message shows that your installation appears to be 
working correctly.
	
To generate this message, Docker took the following steps:
1. The Docker client contacted the Docker daemon.
2. The Docker daemon pulled the "hello-world" image from 
the Docker Hub. (arm64v8)
3. The Docker daemon created a new container from that image 
which runs the executable that produces the output you are 
currently reading.
4. The Docker daemon streamed that output to the Docker client,
which sent it to your terminal.
\end{lstlisting}

\pagebreak
Zoals te zien in de onderstaande afbeelding, de macOS en Linux ARM versie van Docker is volwaardig en deze werkt zoals het hoort.

\begin{figure}[!h]
	\centering
	\includegraphics[width=\linewidth]{img/docker.png}
	\caption{Docker op het Raspberry Pi- en macOS besturingssysteem}
\end{figure}

Het is pas wanneer een gebruiker Docker probeert te installeren op het Windows on ARM platform dat er problemen ontstaan. De Docker Desktop versie die een gebruiker kan downloaden via de officiële Docker website lijkt op het eerste gezicht zonder problemen te installeren, echter wanneer men deze probeert uit te voeren na de installatie verschijnt er een error-boodschap. Dit bericht geeft aan dat het \textit{virtual machine platform} moet worden ingeschakeld en vervolgens dat WSL 2 moet worden geïnstalleerd \autocite{Docker2021}. Het \textit{virtual machine platform} kan een gebruiker inschakelen via het Windows Features paneel. WSL 2 is echter niet ondersteund op toestellen die geen Hyper-V virtualisatie ondersteuning hebben . Deze Windows hypervisor is echter enkel beschikbaar op x86-64 systemen met Intel of AMD processoren of op Windows on ARM toestellen met een Snapdragon processor. De virtuele aanpak die in dit scenario wordt gebruikt maakt echter gebruik van Parallels desktop die draait op een \textit{Apple Silicon} toestel die geen Hyper-V ondersteuning biedt. Windows on ARM toestellen met een Snapdragon processor hebben echter wel Hyper-V ondersteuning en deze toestellen kunnen bijgevolg gebruikmaken van WSL 2 \autocite{Maurer2020}.

\begin{figure}[!h]
	\centering
	\includegraphics[width=\linewidth]{img/docker_issues_winARM.png}
	\caption{Problemen met Docker op het Windows on ARM platform}
\end{figure}

\pagebreak
\subsection{Computerterminal}
ICT-professionals en studenten gebruiken de terminal van hun desktop-besturingssysteem om bepaalde taken efficiënter uit te voeren dan met een grafische gebruikersinterface. Bovendien kunnen sommige geavanceerde operaties enkel met behulp van de terminal worden uitgevoerd. Een computerterminal is een \textit{text-only} venster die bediend wordt door middel van het toetsenbord. Het spreekt voor zich dat het leren van sneltoetsen essentieel is om deskundig gebruik te maken van een terminal. 

Op Linux besturingssystemen zijnde ARM of x86 zijn de GNOME-Terminal, Konsole of xterm de terminals bij uitstek. Welke terminal er standaard onderdeel uitmaakt van het besturingssysteem hangt af van de gebruikte desktopomgeving, maar ze zouden allemaal soortgelijke functionaliteiten en gebruikerservaringen moeten aanbieden. Op macOS is Terminal sinds jaar en dag de standaard terminal terwijl dit bij Windows besturingssystemen de command prompt, Powershell Console en sinds kort, de Windows Terminal zijn \autocite{Woodfine2020}. 

\pagebreak
Alle ARM besturingssystemen die zijn getest voor dit onderzoek, zijnde Raspberry Pi OS, macOS en Windows on ARM, bieden een terminal ervaring aan die niet te onderscheiden valt van hun x86 tegenhangers. In onderstaande afbeelding is te zien hoe er een SSH connectie wordt gemaakt met een lokale server. 

\begin{figure}[!h]
	\centering
	\includegraphics[width=\linewidth]{img/ssh.png}
	\caption{Het maken van een SSH connectie op ARM besturingssystemen}
\end{figure}

\pagebreak
\section{Proof-of-concept Swift applicatie}
Het opzet van deze \textit{proof-of-concept} applicatie is het creëren van een applicatie die uitvoerbaar is op zowel iOS als macOS en dat met een minimum aan overtollige code of verlies aan performantie. Deze opstelling is mogelijk sinds 2020 dankzij de transitie van Intel x86 processoren naar de \textit{Apple Silicon} ARM architectuur \autocite{Apple2020}. Het is tevens belangrijk om te vermelden dat deze platformonafhankelijke manier van applicatie-ontwikkeling enkel mogelijk is op Mac toestellen die gebruik maken van een \textit{Apple Silicon} processor, zijnde de M1, M1 Pro, M1 Max of de M1 Ultra \autocite{AppleDeveloper2022a}. Indien u wenst te controleren of uw toestel gebruik maakt van een \textit{Apple Silicon} processor, dan kan u dit doen via het “Over deze Mac” dialoogvenster. Omwille van privacyredenen is het serienummer in de onderstaande afbeelding onleesbaar gemaakt.

\begin{figure}[h]
    \centering
    \includegraphics[width=100mm, scale=0.5]{img/overdezemac.jpeg}
    \caption{Het macOS 'Over deze Mac dialoogvenster'}
\end{figure}

De Swift programmeertaal is ontwikkeld door Apple en is reeds in gebruik sinds 2014. Deze wordt gebruikt voor het maken van \textit{native} applicaties voor iOS, iPadOS, macOS, tvOS en watchOS \autocite{AppleDeveloper2022b}. \textit{Native} applicaties worden gecompileerd in de machinetaal van het gekozen hardwareplatform. Vandaar dat applicaties die zijn ontwikkeld voor x86 toestellen niet uitvoerbaar zijn op het ARM platform en vice versa \autocite{Gillis2022}. De Swift taal combineert vaak gebruikte elementen uit de C, Objective C en Python programmeertalen en wordt gekenmerkt door eenvoud en performantie. Er wordt algemeen aangenomen dat Swift applicaties eenvoudiger te ontwikkelen zijn dan hun C en Kotlin tegenhangers omwille van de beginners- en gebruiksvriendelijke syntaxis \autocite{AppleDeveloper2022b}.

Apple voorziet uitstekende handleidingen voor zowel beginners als gevorderden, deze zijn tevens gratis verkrijgbaar in de \textit{Apple Books} online boekenwinkel. Verder stelt Apple een rudimentaire \textit{integrated development environment} (IDE) ter beschikking genaamd Swift Playgrounds. Deze is gericht op beginnende programmeurs of kinderen en jongeren die voor het eerst in aanraking komen met software ontwikkeling. Swift Playgrounds is gratis te verkrijgen op het iPadOS en macOS platform \autocite{AppleDeveloper2022b}.

Xcode is de IDE bij uitstek voor Swift ontwikkelaars aangezien deze een goede balans levert tussen nuttige functies en gebruiksgemak. De bijhorende simulator stelt ontwikkelaars in staat om hun applicaties meteen te testen in een gevirtualiseerde omgeving. Ondersteuning voor de \textit{Apple Silicon} architectuur is inbegrepen in Xcode sinds versie 12. Dit onderzoek maakt gebruik van Xcode 13.3.1 en de bijhorende toepassingen \autocite{AppleDeveloper2022a}.

De \textit{proof-of-concept} applicatie omvat een eenvoudige \textit{hard-coded} nieuwsapplicatie, dit houdt in dat de app niet kan bijgewerkt  worden zonder aanpassingen te maken aan de broncode. Verder maakt de app ook geen gebruik van \textit{API calls}. De focus van dit onderzoek ligt namelijk op het onderzoeken van de compatibiliteit en het gebruiksgemak van de applicatie op meerdere platformen en niet op het ontwikkelen van een geavanceerd programma dat gericht is op consumentengebruik.

Vooraleer een Swift ontwikkelaar wenst te starten met een project, is het aangewezen om te controleren of het besturingssysteem en de IDE gebruikmaken van de meest recente softwareversies. In het geval van macOS is dit uitvoering 12.3.1 en bij Xcode is dit versie 13.3.1.

\begin{figure}[!h]
    \centering
    \includegraphics[width=90mm, scale=0.7]{img/xcodeversie.jpeg}
    \caption{Het startscherm van Xcode 13}
\end{figure}

Om te starten met de ontwikkeling van een nieuwe applicatie volstaat het om te kiezen voor de “Create a new Xcode project” optie. Vervolgens verschijnt er een dialoogvenster waar u kan kiezen tussen verschillende sjablonen gaande van iOS applicaties tot programma’s die gericht zijn op het tvOS platform. In het kader van dit onderzoek volstaan de “multiplatform” en “other” sjablonen. Gezien het doel van dit onderzoek, namelijk het onderzoeken van een “multiplatform“ applicatie is, is het aangewezen om te kiezen voor het “other” sjabloon aangezien deze de ontwikkelaar additionele flexibiliteit biedt.

\begin{figure}[!h]
    \centering
    \includegraphics[width=110mm, scale=0.7]{img/otherproject.png}
    \caption{Xcode 13 sjablonen}
\end{figure}

\pagebreak
Na het kiezen van een sjabloon dient een ontwikkelaar een \textit{target} toe te voegen. Een \textit{target} specificeert een te bouwen product en bevat de instructies om dit product te bouwen vanuit een set bestanden in een project- of werkruimte. Een \textit{target} definieert één enkel product, het organiseert de invoer in het bouwsysteem, de bronbestanden en tenslotte de instructies voor het verwerken van de bronbestanden die nodig zijn om het product te bouwen \autocite{AppleDeveloper2011}. Een project kan één of meerdere \textit{targets} bevatten, in het geval van dit project zijn dit macOS en iOS.
\begin{figure}[!h]
    \centering
    \includegraphics[width=110mm, scale=0.7]{img/iostarget.png}
    \caption{Een overzicht van de verschillende targets}
\end{figure}

\begin{figure}[!h]
    \centering
    \includegraphics[width=120mm, scale=0.7]{img/iostargetdetail.png}
    \caption{Een detailweergave van het iOS target}
\end{figure}

\newpage
Eenmaal de iOS en macOS \textit{targets} zijn toegevoegd aan het project, is het tijd om een gedeelde \textit{group} aan te maken die in dit geval de naam “shared” heeft gekregen. In deze gedeelde \textit{group} zullen alle \textit{models} en \textit{views} worden opgeslagen. Deze zijn allen toegankelijk vanuit de iOS en macOS targets. De \textit{shared group} kan vervolgens uitgebreid worden met de \textit{models group} waar alle Swift \textit{structures} terechtkomen. \textit{Structures} zijn gelijkaardig aan klassen in andere programmeertalen, maar toch zijn er enkele opmerkelijke verschillen. Een klasse is een \textit{reference type} terwijl een \textit{struct} een \textit{value type} is. Wanneer men een \textit{struct} kopieert, krijgt men twee unieke kopieën van de gegevens. Echter wanneer men een klasse kopieert, verkrijgt men twee verwijzingen naar één instantie van de gegevens \autocite{Khan2021}. De \textit{Article struct} is als volgt opgebouwd: 

\begin{figure}[!h]
    \centering
    \includegraphics[width=120mm, scale=0.7]{img/articleswift.png}
    \caption{Een overzicht van het initialisatievenster van Article.swift}
\end{figure}

\newpage
\begin{lstlisting}
    
    //  Article.swift
    //  NewsApplication-iOS
    //
    //  Created by Nathan Degryse on 11/05/2022.
    
    import Foundation
    
    enum Subject: String {
        case all = "All"
        case sport = "Sport"
        case internationaal = "Internationaal"
        case regionaal = "Regionaal"
        case politiek = "Politiek"
    }
    
    extension Subject {
        
        var title: String {
            switch self {
                case .all:
                return "All"
                case .sport:
                return "Sport"
                case .internationaal:
                return "Internationaal"
                case .regionaal:
                return "Regionaal"
                case .politiek:
                return "Politiek"
            }
        }
        
    }
    
    struct Article: Identifiable{
        let id = UUID()
        let name: String
        let photo: String
        let description: String
        let rating: Int?
        let subject: Subject
    }
    
\end{lstlisting}
\newpage

Naast de \textit{models group} komt ook de group met alle \textit{preview content} in de \textit{shared group} terecht. Apple verwacht dat developers gebruikmaken van SwiftUI \textit{previews}, dit zijn voorvertoningen die live worden bijgewerkt wanneer een ontwikkelaar nieuwe elementen toevoegt of bestaande elementen aanpast/verwijdert. U kunt de standaard asset-catalogus 'Preview Assets' gebruiken om voorbeeldafbeeldingen, kleuren en andere soorten assets te configureren die u normaal aan een asset-catalogus zou toevoegen. In dit project wordt de asset-catalogus gebruikt om een voorvertoning te genereren van de afbeeldingen die aan bod zullen komen in de nieuwsartikelen \autocite{VanDerLee2021}. Het is essentieel om aan te geven dat de 'Preview Assets' beschikbaar zijn voor beide targets aangezien deze beide gebruik zullen maken van de afbeeldingen.

\begin{figure}[!h]
    \centering
    \includegraphics[width=\linewidth]{img/previewassets.png}
    \caption{Een overzicht van het initialisatievenster van de preview assets}
\end{figure}

\pagebreak 
In deze fase van het project is het cruciaal om de \textit{target membership} van de Article.swift file te verifiëren of aan te passen. Deze dient namelijk beschikbaar te zijn voor zowel het iOS- als macOS \textit{target} aangezien beide applicaties gebruik zullen maken van dit bestand.

\begin{figure}[!h]
    \centering
    \includegraphics[width=80mm, scale=0.5]{img/articletargetmembership.png}
    \caption{Een overzicht van het targetmembership}
\end{figure}

Tenslotte dient de \textit{views group} ook toegevoegd te worden aan de \textit{shared group}. Deze bestaat uit de volgende \textit{views} die gedeeld moeten worden tussen de verschillende \textit{targets}:

\begin{itemize}
    \item \textbf{ArticleListView:} Deze \textit{view} zal gebruikt worden om een lijstvoorstelling te genereren op basis van alle beschikbare nieuwsartikelen. Deze lijst neemt het volledige scherm in beslag op de iOS applicatie, maar neemt echter slechts een deel van het scherm in beslag op de macOS applicatie. Dit omwille van de extra schermruimte die ter beschikking is op grotere toestellen zoals laptops en desktops.
    \item \textbf{ArticleDetailView:} Deze \textit{view} wordt gebruikt om een detailweergave voor te stellen van het desbetreffende artikel. Op het iOS platform neemt dit het volledige scherm in beslag terwijl dit op het Mac platform slechts een deel confisqueert. 
    \item \textbf{RatingView:} Tenslotte zal deze \textit{view} een grafische voorstelling weergeven van de recensie van het betreffende artikel. 
\end{itemize}

Het is opnieuw essentieel om beide \textit{targets} toe te voegen aan alle \textit{views}.

\pagebreak
Eenmaal alle \textit{views} zijn geïmplementeerd in de gedeelde \textit{group} dient er gerefereerd te worden naar deze vanuit de \textit{target} producten zijnde de iOS en macOS applicaties. In onderstaande afbeelding is te zien hoe dit wordt verwezenlijkt in Swift. Via een \textit{navigation view} kan een gebruiker navigeren doorheen een verzameling binnen een navigatie-gebaseerde applicatie. Gebruikers navigeren naar een weergave door middel van een \textit{NavigationLink} te selecteren die de ontwikkelaar verstrekt. Op iPadOS en macOS verschijnt de inhoud van de bestemming in de volgende beschikbare kolom. Op andere platformen zoals iOS en watchOS wordt een nieuwe weergave op de stapel geplaatst en kunnen items uit de stapel worden verwijderd met platform specifieke besturingselementen, zoals een terug-knop of een veegbeweging \autocite{AppleDeveloper2022c}.

\begin{figure}[!h]
    \centering
    \includegraphics[width=\linewidth]{img/contentview.jpg}
    \caption{Een weergave van de ContentView files}
\end{figure}

\pagebreak
In de onderstaande afbeelding is er tenslotte een overzicht beschikbaar van de finale programmastructuur. Deze omvat de iOS en macOS nieuwsapplicaties die gebruikmaken van de gedeelde \textit{views}, \textit{models} en \textit{preview assets}.

\begin{figure}[!h]
    \centering
    \includegraphics[width=70mm, scale=0.7]{img/applicatiestructuur.png}
    \caption{Een systematisch overzicht van de applicatiestructuur}
\end{figure}

\pagebreak
Deze finale afbeelding biedt een weergave van de iOS en macOS applicaties die simultaan uitvoerbaar zijn. Deze maken gebruik van simulatiesoftware die een onderdeel is van Xcode 13. Nadat een ontwikkelaar een applicatie heeft ontwikkeld, kan deze gecompileerd worden en kan de ontwikkelaar deze uitvoeren en testen op een gesimuleerd of een echt apparaat. Een voordeel van de gesimuleerde aanpak is dat een ontwikkelaar de applicatie kan testen op een uitgebreid gamma van virtuele apparaten gaande van de iPod Touch tot de meest recente iPhone modellen \autocite{AppleDeveloper2021}. De volledige broncode van dit project is beschikbaar via onderstaande URL: \url{https://github.com/DegryseNathan/ARM-van-mobile-naar-desktop/tree/main/NewsApplication}
\\\\
\begin{figure}[!h]
    \centering
    \includegraphics[width=\linewidth]{img/iosenmacosapplicatie.png}
    \caption{Een overzicht van de iOS en macOS nieuwsapplicatie}
\end{figure}
